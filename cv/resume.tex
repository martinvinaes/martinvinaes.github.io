% Don't like 10pt? Try 11pt or 12pt
\documentclass[10pt]{article}

% This is a helpful package that puts math inside length specifications
\usepackage{calc}
\usepackage{comment}
\usepackage{mathptmx}% ads time new roman
\usepackage{bibentry}	
\usepackage{microtype} % giver optisk "lige bagkant", bedre orddelingen
\usepackage[english]{babel}
\usepackage[utf8]{inputenc}
%orddeling
\hyphenation{regres-sions-dis-kontinuitets-designet}

% Simpler bibsection for CV sections
% (thanks to natbib for inspiration)
\makeatletter
\newlength{\bibhang}
\setlength{\bibhang}{1em} %1em}
\newlength{\bibsep}
 {\@listi \global\bibsep\itemsep \global\advance\bibsep by\parsep}
\newenvironment{bibsection}%
        {\begin{enumerate}{}{%
%        {\begin{list}{}{%
       \setlength{\leftmargin}{\bibhang}%
       \setlength{\itemindent}{-\leftmargin}%
       \setlength{\itemsep}{\bibsep}%
       \setlength{\parsep}{\z@}%
        \setlength{\partopsep}{0pt}%
        \setlength{\topsep}{0pt}}}
        {\end{enumerate}\vspace{-.6\baselineskip}}
%        {\end{list}\vspace{-.6\baselineskip}}
\makeatother

% Layout: Puts the section titles on left side of page
\reversemarginpar

%\textendash 

\renewcommand{\labelitemi}{$\textendash$}
\renewcommand{\labelitemii}{$\textopenbullet$}

%% Use these lines for letter-sized paper
\usepackage[paper=letterpaper,
            %includefoot, % Uncomment to put page number above margin
            marginparwidth=1.2in,     % Length of section titles
            marginparsep=.05in,       % Space between titles and text
            margin=1in,               % 1 inch margins
            includemp]{geometry}

%% More layout: Get rid of indenting throughout entire document
\setlength{\parindent}{0in}

\usepackage[shortlabels]{enumitem}
\usepackage{etaremune}
\usepackage{fancyhdr,lastpage}
\pagestyle{fancy}
%\pagestyle{empty}      % Uncomment this to get rid of page numbers
\fancyhf{}\renewcommand{\headrulewidth}{0pt}
\fancyfootoffset{\marginparsep+\marginparwidth}
\newlength{\footpageshift}
\setlength{\footpageshift}
          {0.5\textwidth+0.5\marginparsep+0.5\marginparwidth-2in}
\lfoot{\hspace{\footpageshift}%
       \parbox{4in}{\, \hfill %
                    \arabic{page} of \protect\pageref*{LastPage} % +LP
%                    \arabic{page}                               % -LP
                    \hfill \,}}

% Finally, give us PDF bookmarks
\usepackage{color,hyperref}
\definecolor{darkblue}{rgb}{0.0,0.0,0.3}
\hypersetup{colorlinks,breaklinks,
            linkcolor=darkblue,urlcolor=darkblue,
            anchorcolor=darkblue,citecolor=darkblue}

\newcommand{\makeheading}[2][]%
        {\hspace*{-\marginparsep minus \marginparwidth}%
         \begin{minipage}[t]{\textwidth+\marginparwidth+\marginparsep}%
             {\large \bfseries #2 \hfill #1}\\[-0.15\baselineskip]%
                 \rule{\columnwidth}{1pt}%
         \end{minipage}}

% The section headings
%
% Usage: \section{section name}
\renewcommand{\section}[1]{\pagebreak[3]%
    \hyphenpenalty=10000%
    \vspace{1.3\baselineskip}%
    \phantomsection\addcontentsline{toc}{section}{#1}%
    \noindent\llap{\scshape\smash{\parbox[t]{\marginparwidth}{\raggedright #1}}}%
    \vspace{-\baselineskip}\par}

% An itemize-style list with lots of space between items
\newenvironment{outerlist}[1][\enskip\textbullet]%
        {\begin{itemize}[#1,leftmargin=*]}{\end{itemize}%
         \vspace{-.6\baselineskip}}

% An environment IDENTICAL to outerlist that has better pre-list spacing
% when used as the first thing in a \section
\newenvironment{lonelist}[1][\enskip\textbullet]%
        {\begin{list}{#1}{%
        \setlength{\partopsep}{0pt}%
        \setlength{\topsep}{0pt}}}
        {\end{list}\vspace{-.6\baselineskip}}

% An itemize-style list with little space between items
\newenvironment{innerlist}[1][\enskip\textbullet]%
        {\begin{itemize}[#1,leftmargin=*,parsep=0pt,itemsep=0pt,topsep=0pt,partopsep=0pt]}
        {\end{itemize}}

% An environment IDENTICAL to innerlist that has better pre-list spacing
% when used as the first thing in a \section
\newenvironment{loneinnerlist}[1][\enskip\textbullet]%
        {\begin{itemize}[#1,leftmargin=*,parsep=0pt,itemsep=0pt,topsep=0pt,partopsep=0pt]}
        {\end{itemize}\vspace{-.6\baselineskip}}

% To add some paragraph space between lines.
% This also tells LaTeX to preferably break a page on one of these gaps
% if there is a needed pagebreak nearby.
\newcommand{\blankline}{\quad\pagebreak[3]}
\newcommand{\halfblankline}{\quad\vspace{-0.5\baselineskip}\pagebreak[3]}

% Uses hyperref to link DOI
\newcommand\doilink[1]{\href{http://dx.doi.org/#1}{#1}}
\newcommand\doi[1]{doi:\doilink{#1}}

% For \url{SOME_URL}, links SOME_URL to the url SOME_URL
\providecommand*\url[1]{\href{#1}{#1}}
% Same as above, but pretty-prints SOME_URL in teletype fixed-width font
\renewcommand*\url[1]{\href{#1}{\texttt{#1}}}


%remove first blank page
\begin{document}
%\nobibliography{cvpubs} 
%\bibliographystyle{chicago}
%\newpage
%\pagenumbering{Roman}
%\setcounter{2}{1}

%\pagestyle{empty}
%Bilag B
%CONTACT INFORMATION*****************************************************************************************************************
\begin{large} \bf Martin Vin\ae s Larsen \end{large}  $\cdot$   \today  	
 \vspace{1em} 
			
\section{Contact}
			
Department of Political Science\hfill Office: (+45) 35 32 46 37 \\
University of Copenhagen   \hfill Mobile: (+45) 60 71 65 45  \\
{\O}ster Farimagsgade 5, opgang E \hfill Work e-mail: mvl@ifs.ku.dk \\
DK-1353 Copenhagen K, Denmark  \hfill Private e-mail: vinaes@gmail.com \\
																		
%RESEARCH INTERESTS*******************************************************************************************************************
\section{Research Interests}

Electoral accountability and the quality of voter decision-making, economic voting, rational choice theory, political psychology, the role of information in EU elections and causal inference. 


%EMPLOYMENT*******************************************************************************************************************************
%\section{Academic Appointments}

%{\bf Department of Political Science, University of Copenhagen:}  \hfill  Nov. 2013 -- \\
%$\cdot$ Assistant professor (public administration and quantitative methods) \\

%{\bf Danish Ministry of Education: } \hfill Nov. 2013 --  \\
%$\cdot$ National external examiner corps for political science  \\

%{\bf Department of Political Science, University of Copenhagen:}   \hfill May 2013 --  \\
%$\cdot$ Post.doc. \\
		
%{\bf Department of Political Science, University of Copenhagen:}  \hfill Feb. 2013 -- Apr. 2013 \\
%$\cdot$ Scientific employee \\

%{\bf University of Copenhagen:} \hfill 2008 -- 2009  \\
%$\cdot$ Student research assistant \\

%{\bf Department of Political Science, University of Copenhagen:} \hfill 2007 -- 2008 \\
%$\cdot$  Student research assistant


%EDUCATION*******************************************************************************************************************************
\section{Education}
		
{\bf Department of Political Science, University of Copenhagen:} \hfill Aug. 2017 (expected) \\
$\cdot$ PhD: ``Electoral accountability and Economic voting''. \\
$\cdot$ Advisor: Professor Kasper M\o ller Hansen.  \\

{\bf Department of Political Science, University of Copenhagen:} \hfill Aug. 2015 \\
$\cdot$ M.Sc. Political science \\
$\cdot$  Thesis: ``Voters infer incumbent quality from the state of the economy.''  \\

{\bf Chair in Electoral democracy, University of Montreal:}  \hfill Aug. 2014 -- Jan. 2015  \\
$\cdot$ Visiting researcher \\
	
{\bf Department of Political Science, University of Copenhagen:} \hfill Jan. 2012\\ 
$\cdot$ B.Sc. Political science \\



%Employment

\section{Additional training}
{\bf ICPSR, University of Michigan:} \hfill Aug 2014 \\
$\cdot$ Summer Program \\

{\bf Experiments, University of Oxford:} \hfill July 2012 \\
$\cdot$ Summer Program \\

{\bf Summer Institute in Political Psychology, Stanford University:}, \hfill Aug. 2011 \\
$\cdot$ Summer Program \\
\section{Employment}

{\bf Department of Political Science, University of Copenhagen:} \hfill Aug. 2013  ---  \\
$\cdot$ Ph.D. Fellow\\



{\bf EUvox:} \hfill Mar. 2014 \\
$\cdot$ Communications director for Danish part of the EUvox VAA\\

{\bf Ministry of Finance:} \hfill Dec. 2011 --- Jan. 2013 \\
$\cdot$ Student assistant, Office of Municipal Economics and Regions \\

{\bf Department of Political Science, University of Copenhagen:} \hfill Dec. 2010 --- Jan. 2013 \\
$\cdot$ Teaching  Assistant in Research Methods I \\


\newpage

%PUBLICATIONS****************************************************************************************************************************
%\renewcommand\refname{\bibnumfmt
%enumerate should decrease

\section{Articles with revise \& resubmit}
\begin{etaremune}
%PEER REVIEWED ****************

\item \textbf{Larsen, Martin Vinæs} (2016). Incumbent Tenure Crowds Out Economic Voting. \textit{Revise \& Resubmit in The Journal of Politics}.

\item Dahlgaard, Jens Olav; Hansen, Jonas H; Hansen, Kasper M; \textbf{Larsen, Martin Vinæs} (2016) How Election Polls Shape Voting Behavior \textit{Revise \& Resubmit in Scandinavian Political Studies}.

\hspace{-1cm}\section{Peer Reviewed Articles}


\item Beach, Derek; Hansen, Kasper M.; \textbf{Larsen, Martin Vinæs} (2016). How campaigns enhance European issues voting during European Parliament elections.  \textit{Political Science Research and Methods}, Conditional Accept.

\item Dahlgaard, Jens Olav; Hansen, Jonas H; Hansen, Kasper M; \textbf{Larsen, Martin Vinæs} (2016). How are Voters Influenced by Opinion Polls? The Effect of Polls on Voting Behavior and Party Sympathy  \textit{World Political Science Review}, Early View.

\item Esben Høgh; \textbf{Larsen, Martin Vinæs} (2016). Can information increase turnout in European Parliament elections? Evidence from a quasi-experiment in Denmark. \textit{JCMS: Journal of Common Market studies.} 54 (6), 1495---1508. 

\item \textbf{Larsen, Martin Vinæs} (2016). Economic Conditions Affect Support for Prime Minister Parties in Scandinavia. \textit{Scandinavian Political Studies.} 39 (3), 226--–241. 

\item Mats Joe Bordacconi; \textbf{Larsen, Martin Vinæs} (2014). Regression to causality: Regression-style presentation influences causal attribution. \textit{Research and politics.} 1 (2).

\item Nielsen, Sigge Winther; \textbf{Larsen, Martin Vinæs} (2014). Party Brands and Voting. \textit{Electoral Studies.} 33, 153---165.


% \hspace{-1cm}\section{Comments and Discussions}


%CONFERECE and WORKING PAPER ****************
\hspace{-1cm}\section{Conference and Working papers}

\item \textbf{Larsen, Martin Vinæs}; Olsen, Asmus Leth (2015). Incumbent tax setting and clarity of political responsibility. \textit{Presented at the American Political Science Association Meeting, 2016, panel: ``Economic foundations of electoral politics''}. 

\item \textbf{Larsen, Martin Vinæs} (2015). Is there a self-serving bias in attribution of political responsibility? \textit{Presented at the Midwest Political Science Association Meeting, 2016, Panel: ”Accountability for Economic Performance”.}

\item \textbf{Larsen, Martin Vinæs}; Frederik Hjorth; Kim Sønderskov; Peter Dinesen (2015). Housing Bubbles and Incumbent Support. \textit{American Political Science Association Meeting, 2016, Panel: New Directions in Policy Feedback Research}.

\item \textbf{Larsen, Martin Vinæs}  (2015). Clarity of responsibility and electoral accountability: Evidence of a causal relationship from reform of labor market regulation. \textit{Presented at the Midwest Political Science Association Meeting, 2015, panel: ``Experimental Approaches to Economic Perceptions''}.


%DANISH PEER REVIEWED ****************

\hspace{-1cm}\section{Danish Peer Reviewed Articles}

\item Dahlgaard, Jens Olav; Hansen, Jonas H; Hansen, Kasper M; \textbf{Larsen, Martin Vinæs} (2015). Hvordan påvirkes vælgerne af meningsmålinger? Effekten af meningsmålinger på danskernes stemmeadfærd og sympati for partierne. \textit{Politica} 46 (1): 5-23

\item Nielsen, Sigge Winther; \textbf{Larsen, Martin Vinæs}; Egeberg, Jesper de Hemmer  (2011). Vi er alle spindoktorer. \textit{{\O}konomi og Politik} 84 (1): 19--33

\newpage


%BOOKS ****************
\hspace{-1cm}\section{Book chapters}

\item \textbf{Larsen, Martin Vinæs}; Nielsen, Sigge Winther (2016). Den kommenterede valgkamp, in Goul Andersen, Jørgen; Shamshiri, Ditte: \textit{Vælgere på Vandring}. Frederiksberg: Frydenlund Academic.

\item Nielsen, Sigge Winther; \textbf{Larsen, Martin Vinæs}; Egeberg, Jesper de Hemmer (2012). Den politiske mandagstræner, in Nielsen, Sigge Winther: Politisk Marketing. København: Karnov Group.

\item \textbf{Larsen, Martin Vinæs} (2012). Praktikerne har ordet: Samtaler med de danske partier, in Nielsen, Sigge Winther: Politisk Marketing. København: Karnov Group.

%REPORTS ****************
\hspace{-1cm}\section{Reports}

\item \textbf{Larsen, Martin Vinæs} (2014). Gennemsnit af meningsmålinger forudsiger
valgresultatet. CVAP Working Paper Series 2016.

\item Dahlgaard, Jens Olav; Hansen, Jonas H.; Hansen, Kasper M.; \textbf{Larsen, Martin Vinæs} (2014). Hvordan påvirkes vælgerne af meningsmålinger? Rapport om effekten af meningsmålinger på danskernes stemmeadfærd og sympati for partierne. CVAP Working Paper Series 1/2014.





%REVIEWS ****************
 

\end{etaremune}


\hspace{-1cm}\section{Work in progress}


\begin{itemize}

\item ``Translating Preferences into Numbers (with Rasmus Tue Pedersen)'' 

\item ``Benchmarking responsibly? How voters compare economic performance.'' 

\item ``Who is poor? People's perception of poverty (with Esben Hogh).'' 

\item ``Absolute Majorities as a Regression Discontinuity: Applications and Issues'' (with Asmus Leth Olsen) 





\end{itemize}

%Professional Memberships:
%- American Political Science Association (2011-)
%- Midwest Political Science Association (2011–)
% PMRC?

%ACADEMIC TEACHING EXPERIENCE*********************************************************************************************************
\section{Academic Teaching}

Teaching at the Department of Political Science, University of Copenhagen: \\




\textbf{M.Sc.-level course:}

\begin{itemize}

\item ``New and classic approaches to Political Behavior''\\
$\cdot$ Course design and course teaching: spring 2015.	

\item ``Advanced Quantitave methods''\\
$\cdot$ Course design and course teaching: spring 2015.	

\end{itemize}

\textbf{BA.-level course:}

\begin{itemize}

\item ``Theories and Approaches to Political Science'' \\
$\cdot$ Full course teaching spring 2014, lectures in fall, 2014, 2015, 2016.	

\item ``Advanced Quantitave methods''\\
$\cdot$ Course design and course teaching in spring 2015.	
	
\item ``Research methods 2'' \\
$\cdot$ Ran workshops in spring of 2014.

\item ``Research methods 1'' \\
$\cdot$ TA in fall of 2010, 2011, 2012 and 2013.

\end{itemize}

%DEPARTMENTAL SERVICE**************************************************************************************************

%ACADEMIC TEACHING EXPERIENCE*********************************************************************************************************
\section{Academic Supervision}
Supervised 4 MA-thesis (6 students). 


%AWARDS*********************************************************************************************************
%\section{Awards}

%$\cdot$ \textit{Teacher of the Year 2015} at the Ms.C.-level at the Department of Political Science in Copenhagen (academic year 2014/2015). Awarded the prize for my course \textit{The Psychology of the Public Sector} taught fall of 2014.  \\




%GRANTS*********************************************************************************************************
\section{Grants and Awards}

Best Paper in Politica 2015. \\

 Nominated for \textit{Teacher of the Year 2012} and \textit{Teacher of the Year 2015} at the Department of Political Science, University of Copenhagen for courses \textit{Research Methods 1} and \textit{Political behavior}. \\

Travel grants:
$\cdot$ Oticon foundation: \underline{19,500 DKK}, 2014, 2015. \\

$\cdot$ Knud Højgaards foundation: \underline{22,000 DKK}, 2014, 2015. \\
  
$\cdot$ Augustinus foundation: \underline{20,000 DKK}, 2014.   \\

$\cdot$ Danish Tennis Foundation: \underline{10,000 DKK}, 2014. \\ 


%INVITED TALKS*********************************************************************************************************
\section{Talks and Conference Presentations}

Paper presentations at conferences: American Political Science Association Conference 2015, 2016; Midwest Political Science Association 2015, 2016 (2); Amsterdam Political Psychology Conference 2015; Danish Political Science Association 2015, 2016 (2). \\

Clarity of Responsibility and economic voting. Presentation at University Laval, Canada. \textit{27-11-2014} \\

Economic voting and time in office. Presentation at the Chair in Electoral Studies, University of Montreal, Canada. \textit{12-02-2014}  \\



\section{Service to the profession and administration}

$\cdot$ Reviewer experience: Electoral Studies, Scandinavian Political Studies, European Union Politics, Political Behavior. \\

Discussant: Danish Political Science Association: 2015, 2016. \\

Chair: Danish Political Science Association: 2015. \\

Ph.D. representative, Department of Political Science, University of Copenhagen, 2014 -- 2016  \\

Ph.D. club president, Department of Political Science, University of Copenhagen, 2016 --  \\


\section{Media}

Regular commentator on political events, various news platforms: radio, TV, regional and national Danish newspapers ($>100$ times a year since 2013).  \\ \vspace{0.1in}

 Research results featured in: 
 $\cdot$ P1 Orientering 
 $\cdot$ Ugebrevet A4
 $\cdot$ Videnskab.dk
 $\cdot$ Berlingske
 $\cdot$ Politiken 
 $\cdot$ Information 
 $\cdot$ Jyllands Posten
 $\cdot$ DR2 Deadline 

\section{Political Writing}
{\bf PoliLab, Politiken:} \hfill Sep. 2015  -- Jun. 2016  \\
$\cdot$ Contributer to PoliLab with articles about politics and political research.\\

{\bf Analysis, tv2.dk:} \hfill May 2014  -- Jun. 2015  \\
$\cdot$ Contributor to tv2.dk with articles about politics and political research.\\
\end{document}

